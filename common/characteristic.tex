
{\actuality} С момента открытия первичного космического излучения (ПКЛ) прошло уже более 100 лет, однако вопросы связанные с механизмами генерации и распространения этих частиц до сих пор остаются актуальными. Космические лучи, впервые открытые В. Гессом 17 апреля 1912 \cite{VHess} года позволили существенно продвинуться в пониманиии физики элементарных частиц. В частности, именно в космических лучах были открыты такие элементарные частицы как позитроны, мюоны, $\pi$-мезоны, гипероны и К-мезоны. Последние привели к постулированию нового квантового числа \textquotedbl странности\textquotedbl{} и в конечном счете экспериментальному доказательству несохранения P-четности в слабых взаимодействиях\cite{Wu}. Кроме того, эксперименты с космическими лучами позволили установить процессы множественной генерации частиц в нуклонных взаимодействиях и определить величину эффективного сечения взаимодействия нуклонов высокой энергии.

В настоящее время под первичными космическими лучами как правило понимают протоны и ядра атомов галактического и внегалактического происхождения с энергиями от $\~10^6$эВ до $\~10^{21}$эВ. Спектр ПКЛ быстро падает с ростом энергии по закону  $F(E) \sim E^{-a}$ где $a$ меняется от 2.7 до 3.2 в зависимости от энергетической области. Поскольку ПКЛ обладают электрическим зарядом, их траектории в космическом пространстве искривляются магнитными полями различных объектов, что приводит к тому, что в околоземном пространстве регистрируется однородный и изотропный поток ПКЛ. В связи с этим до сих пор актуальным остается вопрос происхождения космических лучей. 

Наибольших успехов в решении данного вопроса к настоящему времени удалось достичь посредством гамма-астрономии - раздела астрономии, изучающего объекты по их гамма-излучению. Поскольку фотоны не имеют электрического заряда, направление прихода таких частиц указывает на источник в котором они рождаются. Следовательно, детектирование гамма-квантов позволяет изучать процессы происходящие в удаленных объектах метагалактики и косвенным образом свидетельствовать об ускорении в них космических лучей до предельно высоких энергий.

Эпоха гамма-астрономии началась в середине прошлого века, когда группой А.Е.Чудакова были проведены первые попытки детектирования гамма-квантов посредством регистрации Широких Атмосферных Ливней. Как для гамма-квантов, так и для космических лучей атмосфера земли непрозрачна. Частицы, попадая в верхние слои атмосферы, взаимодействуют с молекулами среды, вызывая лавинообразное увеличения числа вторичных частиц, движущихся вдоль траектории первичной частицы. Данный процесс называется Широким Атмосферным Ливнем (ШАЛ) и может быть зарегистрирован у поверхности Земли.

Несмотря на то, что в эксперименте Чудакова не удалось получить значимого результата, попытки косвенной регистрации гамма-квантов продолжились и в 1960 году Д. Коккони опубликовал работу, посвященную успешной регистрации гамма-квантов с энергией $\sim10^{12}$эВ от Крабовидной туманности. 

Гамма-астрономия является одним из наиболее активно развивающихся разделов астрофизики. Создаваемые в настоящее время экспериментальные установки способны регистрировать потоки космических лучей составляющие доли процента от потока Крабовидной туманности - самого яркого в гамма диапазоне источника известного на данный момент. Помимо вопроса о происхождении космических лучей, гамма-астрономия также нацелена на исследование процессов, протекающих в источниках излучения и механизмов распространения гамма-квантов в космическом пространстве, а также поиски новой физики, в частности темной материи.

В последнее время, наибольших успехов в детектировании гамма-квантов предельно высоких энергий добились обсерватории LHASSO и $Tibet-AS\gamma$, в которых реализуется выделение гамма-квантов из фона космических лучей по числу мюонов во вторичной компоненте ШАЛ. В 2019 году $Tibet-AS\gamma$ опубликовал работу, посвященную детектированию гамма-квантов от Крабовидной туманности с энергиями выше 100 ТэВ\cite{Tibet2019}.  Полученный $Tibet-AS\gamma$ результат косвенно указывает на то что в данном источнике происходит ускорение частиц до ПэВного порядка энергий. Оценки магнитного поля Крабовидной туманности указывают на то что ускоряются до ПэВа по всей видимости электроны, и вопрос о происходении космических лучей все еще остается открытым. Тем не менее, открытие $Tibet-AS\gamma$ привело к оживлению интереса теоретиков к процессам генерации частиц в Крабовидной туманности, а также показало что современные эксперименты способны работать в данной энергетической области. Так, гамма-обсерватория LHASSO в 2023 году выпустила первый каталог источников гамма-квантов, содержащий 43 объекта, от которых были зарегистрированы гамма-кванты с энергией выше 100 ТэВ\cite{Lhaaso_cat2023}.

Поскольку в основе работы LHASSO и $Tibet-AS\gamma$ лежит одна и та же методика детектирования ШАЛ, представляется интересным наблюдение 100 ТэВных гамма-квантов от зарегистрированных ими объектов посредством другой методики.   

%Данный источник уже долгое время представлялся исследователям как хорошо изученный, а новые обсерватории использовали его в качестве "стандартной свечи" - объекта на котором отрабатывается методика работы и верификация результатов.

Атмосфера Земли для ПКЛ непрозрачна, поэтому для прямого детектирования ПКЛ используются орбитальные детекторы [DAMPE, CALET]. В энергетической области выше нескольких десятков ТэВ поток космических лучей составляет 1 частица на квадратный метр в час, в связи с чем прямые измерения потока ПКЛ оказываются трудно реализуемыми. На смену им приходит косвенный метод регистрации космических лучей, основанный на том что при попадании высокоэнергичных ПКЛ в атмосферу Земли запускается процесс лавинообразного рождения вторичных частиц, движущихся к земле вдоль траектории первичной частицы. Данный процесс называется Широким Атмосферным Ливнем (ШАЛ) и может быть зарегистрирован у поверхности Земли. 

%ШАЛ возникает из-за ядерного взаимодействия первичной частицы с молекулами атмосферы. В результате этого взаимодействия рождаются частицы адронной компоненты ШАЛ: протоны, нейтроны, $\pi^\pm$, $K^\pm$. $\pi^\pm$ и $K^\pm$ могут распадаться, в результате чего рождаются мюоны и нейтрино, которые обладают высокой проникающей способностью и могут быть зарегистрированны даже под слоем грунта. Также, в результате распада $\pi^0$ и $\mu^\pm$ рождаются гамма-кванты и электроны, относящиеся к лекой компоненте ШАЛ. Помимо частиц, распространение ШАЛ сопровождается излучением атмосферы: флуоресцентным и черенковским.

Для детектирования ШАЛ на земле можно использовать как регистрацию заряженных частиц, так и черенковского или флуоресцентного излучения. В настоящее время подавляющее большинство обсерваторий используют комбинированный метод - детектирование нескольких компонент ШАЛ.

Наряду с физикой космических лучей выделяют также гамма-астрономию, изучающую обьекты по их гамма-излучению. Характерные энергии гамма-квантов лежат в области от сотен кэВ до 1 ПэВ. Поскольку фотоны не имеют электрического заряда, направление прихода таких частиц указывает на источник в котором они рождаются. Следовательно, детектирование гамма-квантов позволяет изучать процессы происходящик в удаленных объектах метагалактики и косвенным образом свидетельствовать о ускорении в них космических лучей.

Для гамма-квантов, так же как и для ПКЛ атмосфера Земли непрозрачна, в связи с чем орбитальные обсерватории ограниченны наблюдениями гамма-квантов с энергией не более нескольких ТэВ. Наземная же регистрация ШАЛ от гамма-квантов также возможна и успешно применяется рядом обсерваторий []. Однако, при таком подходе возникает сложность, связанная с тем, что ШАЛ от гамма-квантов требуется выделить на фоне гораздо более многочисленных ШАЛ от ПКЛ. Так, при энергии в 1 ТэВ, на 1 гамма-квант приходится $10^4$ протонов.

В настоящее время существует два успешно реализуемых подхода к режекции фона ПКЛ:
\begin{enumerate}[beginpenalty=10000] % https://tex.stackexchange.com/a/476052/104425
	\item  по числу мюонов во вторичной компоненте;
	\item  по "компактности" изображения ШАЛ.
\end{enumerate}
Под "компактностью" в данном случае подразумевается пространственное распределение частиц или излучения ШАЛ в плоскости перпендикулярной направлению распространения ливня. В связи с этим второй подход имеет несколько вариантов реализации:
\begin{enumerate}[beginpenalty=10000] % https://tex.stackexchange.com/a/476052/104425
	\item  детектирование черекновского излучения ШАЛ;
	\item  регистрация заряженной компоненты ливня.
\end{enumerate}
В последнее время, наибольших успехов в детектировании гамма-квантов предельно высоких энергий добились обсерватории LHASSO и $Tibet-AS\gamma$, в которых реализуется выделение гамма-квантов по числу мюонов во вторичной компоненте ШАЛ. Преимущество данной методики состоит в том что время работы обсерватории слабо зависит от погодных условий и наблюдения могут проводиться практически круглогодично. Наблюдения же посредством детектирования черенковского света ШАЛ ограниченно ясными безлунными ночами. Однако, данный метод оказывается достаточно эффективным при применении Атмосферных черенкоских телескопов (IACT - Imaging Athmospheric Cherenkov Telescope). Телескоп представляет собой зеркало на аль-азимутальной платформе, в фокусе которого расположена камера. Угол обзора классических IACT составляет 1-3 градуса, что позволяет проводить наблюдения за отдельными источниками гамма-излучения. Адронный фон при этом оказывается значительно подавленным благодаря малому углу обзора телескопа, а эффективная площадь может достигать 0.5 $\text{км}^2$\cite{ASTRI2013}. Еще большей эффективности удается достичь при использовании нескольких IACT, расположенных на расстоянии \~100м друг от друга. Так, один и тот же ШАЛ может быть зарегистрирован сразу несколькими телескопами установки, а параметры первичной частицы будут восстановленны с большей точностью.


Наличие же в объекте ускоренных протонов и ядер ведет к тому что некотороя часть из них будет взаимодействовать с окружающей средой, порождая  $\pi^\text{0}$ мезоны, которые в последствии будут распадаться на $2\gamma$.

Наличие в объектре ПКЛ и фотонов в широком энергетическом диапазоне 

Благодаря появлению гамма-астрономии, вопрос об истониках ПКЛ 

Нижняя граница спектра космических лучей обусловлена присутствием разнородных магнитных полей в космическом пространстве, благодаря которым такие частиц оказываются быстро захваченными сразу после генерации и их дальнейшее движение модулируется магнитным полем того или иного объекта. Кроме того, существенная часть низкоэнергичных космических лучей ($E < 10^{11}$эВ), регистрируемых в околоземном пространстве имеет солнечную природу и классифицируется как Солнечные Космические Лучи (СКЛ).

Природа верхней энергетической границы спектра космических лучей до сих пор остается предметом активных исследований. В частности, существует два наиболее реалистичных обяснения обрезния энергетического спектра ПКЛ:
\begin{enumerate}[beginpenalty=10000] % https://tex.stackexchange.com/a/476052/104425
	\item Предел Грайзена-Зацепина-Кузьмина. Эффект возникает вследстивии роста сечения рекции взимодействия протовов с энергией выше  $5*10^{19}$эВ с фотонами косического микроволнового фона (Cosmic Microwave Bakground - CMB). Средняя длинна поглощения протонов при этом составляет 50Мпс, а поскольку в данном радиусе источников ПКЛ не обнаружено, все протоны должны эффективно взаимодейстовать с CMB и порождать  $\pi^\text{+}$ и $\pi^\text{0}$. Последние в конечном счете должны приводить к возникновению диффузного гамма-излучения в результате реакции $\pi^\text{0} -> 2\gamma$.
	\item Предел ускорения ПКЛ в источниках. Максимальная энергия до которой частица может быть ускорена в источнике выражается критерием Хилласа: $\varepsilon_{max} = qBR$, где q - заряд частицы, B - магнитное поле и R - размер ускорителя. Таким образом, согласно критерию Хилласа, протоны могут быть ускорены до энергий $\~10^{20}$эВ, а ядра железа в том же источнике получат на порядок большую энергию.
\end{enumerate}

 Основная проблема изучения данной энергетической области связана с тем, что поток космических лучей падает с ростом энергии пропорционально $\~E^{-a}$ где $a$ меняется от 2.7 до 3.2 в зависимости от энергетической области. Таким образом число частиц с энергией выше $\~3*10^{18}$эВ падающих на площадь $1\text{км}^2$ в год составляет 5-10 штук. Набор статистически значимого количества экспериментальных данных при этом требует покрытия огромной площади порядка нескольких сотен и даже тысяч квадратных километров.

Очевидно, что прямое детектирование первичного космического излучения (ПКЛ) такой плотности невозможно. По этой причине, для регистрации космических лучей с энергией выше $10^9$эВ используются наземные детекторы, детектирующие вторичное космическое излучение, возникающее при взаимодействии ПКЛ с атмосферой. Распространение в атмосфере вторичных космических лучей называется Широким Атмосферным Ливнем (ШАЛ), регистрация которых позволяет косвенно детектировать ПКЛ и проводить исследования космических лучей с энергией выше $10^9$эВ.

Проблема верхней границы спектра космических лучей в настоящее время успешно решается двумя наземными обсерваториями: The Pierre Auger Observatory (PAO), площадью 3000 $\text{км}^2$ в Аргентине и Telescope Array (TA), площадью 700 $\text{км}^2$ в США. Обсерватории ведут наблюдения уже более 15 лет и к настоящему моменту накопили существенную статистику в энергетической области выше $\~3*10^{16}$эВ. Результаты наблюдений показывают, что энергетический спектр ПКЛ имеет экспоненциальное обрезание в районе $\~10^{20} - 10^{21}$эВ, а массовый состав ПКЛ с энергиями выше $\~3*10^{18.7}$эВ имеет тенденцию к утяжелению (преобладают частицы с зарядовым числом Z > 2), что может быть как следствем предела ускорения протонов в источниках, так и комбинацией данного эффекта с пределом ГЗК. Обрезание же спектра ПКЛ вероятнее всего связано с пределом ускорения тяжелых ядер в источниках. 

Исследования физики космических лучей можно разделить на два основных направления:
\begin{enumerate}[beginpenalty=10000] % https://tex.stackexchange.com/a/476052/104425
	\item Физика элементарных частиц;
	\item Физика космоса.
\end{enumerate}
Несмотря на то что эксперименты на ускорителях позволили с огромной точностью проверить стандартную модель, их возможности ограниченны максимальной энергией сталкивающихся пучков. Так, максимально достижимая энергия Большого Адронного Коллайдера (БАК), крупнейшего на данный момент ускорителя элементарных частиц составляет 14ТэВ. В космическом излучении присутствуют частицы с энергией на ~7 порядков больше предела БАК, что говорит о том что во Вселенной существуют гораздо более мощные \textquotedbl коллайдеры\textquotedbl{}, \textquotedbl данные\textquotedbl{} с которых позволяют исследовать физику элементарных частиц в энергетической области пока что недоступной ускорителям.

Исследования космических лучей в контексте физики космоса нацелены на изучение процессов, протекающих в объектах галактики и метагалактики, а также механизмов распространения и взаимодействия ПКЛ с окружающей средой.

На равне с физикой космических лучей выделяют гамма-астрономию и нейтринную астрофизику, поскольку нейтрино и фотоны высоких энергий - гамма-кванты($>10^6$эВ) не отклоняются магнитными полями и несут информацию о направлении на источник. 

Масса покоя нейтрино составляет доли эВ, в связи с чем сечение взаимодействия этих элементарных частиц с веществом крайне мало. Благодаря этому свойству нейтрино рожденые в недрах высокоэнергичных объектов могут покидать его и нести информацию о внутренних процессах ускорителей КЛ. Детектирование нейтрино оказывается очень сложной задачей по той же причине. Для эффективной регистрации нейтрино требуется колоссальный объем детектора. Так, самые крупные существующие в настоящее время нейтринные телескопы IceCube и Baikal-GVD занимают объем 1$\text{км}^2$ и 0.5$\text{км}^2$ соответственно. Кроме того, для детектрования нейтрино требуется прозрачная среда, в которой может распространяться излучение Вавилова-Черенкова - узконаправленое свечение среды, вследствие движения в ней заряженных элементарных частиц со скоростью, превышающей скорость света в этой среде. В связи с этим, IceCube расположен во льдах Антарктиды, а Baikal-GVD в водной толще озера Байкал.

Регистрация гамма-квантов является более простой задачей и решается она как прямыми методами так и косвенными. Так, гамма-кванты с энергиями ниже $10^9$эВ эффективно регистрируются прямым методом орбитальными телескопами, такими как Fermi-LAT. Для детектирования гамма-квантов более высоких энергий, так же как и для космических лучей требуется большая эффективная площадь детектора, что делает орбитальные телескопы не эффективными в этой области и им на смену приходят наземные детекторы, регистрируюшие ШАЛы, индуцируемые гамма-квантами в атмосфере.

ШАЛы детектируются посредством регистрации заряженных частиц (HAWC, TibetAS$\gamma$, LHASSO), а также посредством регистрации черенковского излучения ШАЛ (MAGIC, HESS, VERITAS). Преимущество первого подхода состоит в том что наблюдения могут проводиться независимо от погодных условий и времени суток. Минус данного подхода - поток космических лучей, который в гамма-астрономических наблюдениях является фоном. В области энергий $\~10^{12}$эВ на 1 гамма-квант приходится $10^4$ адронов, подавление которых является достаточно сложной задачей.
Подход при котором регистрируется черенковское излучение ШАЛ реализуем лишь в ясные безлунные ночи, что приводит к несопоставимо более короткой экспозиции по сравнению с обсерваториями заряженных частиц ШАЛ. Однако, при таком подходе удается эффективно подавлять адронный фон посредством ограничения угла обзора телескопа узкой областью вокруг источника. Такой подход позволяет за достаточно короткий промежуток времени получить значимый избыток гамма-квантов над фоном.

Обзор, введение в тему, обозначение места данной работы в
мировых исследованиях и~т.\:п., можно использовать ссылки на~другие
работы~\autocite{Gosele1999161,Lermontov}
(если их~нет, то~в~автореферате
автоматически пропадёт раздел <<Список литературы>>). Внимание! Ссылки
на~другие работы в~разделе общей характеристики работы можно
использовать только при использовании \verb!biblatex! (из-за технических
ограничений \verb!bibtex8!. Это связано с тем, что одна
и~та~же~характеристика используются и~в~тексте диссертации, и в
автореферате. В~последнем, согласно ГОСТ, должен присутствовать список
работ автора по~теме диссертации, а~\verb!bibtex8! не~умеет выводить в~одном
файле два списка литературы).
При использовании \verb!biblatex! возможно использование исключительно
в~автореферате подстрочных ссылок
для других работ командой \verb!\autocite!, а~также цитирование
собственных работ командой \verb!\cite!. Для этого в~файле
\verb!common/setup.tex! необходимо присвоить положительное значение
счётчику \verb!\setcounter{usefootcite}{1}!.

Для генерации содержимого титульного листа автореферата, диссертации
и~презентации используются данные из файла \verb!common/data.tex!. Если,
например, вы меняете название диссертации, то оно автоматически
появится в~итоговых файлах после очередного запуска \LaTeX. Согласно
ГОСТ 7.0.11-2011 <<5.1.1 Титульный лист является первой страницей
диссертации, служит источником информации, необходимой для обработки и
поиска документа>>. Наличие логотипа организации на~титульном листе
упрощает обработку и~поиск, для этого разметите логотип вашей
организации в папке images в~формате PDF (лучше найти его в векторном
варианте, чтобы он хорошо смотрелся при печати) под именем
\verb!logo.pdf!. Настроить размер изображения с логотипом можно
в~соответствующих местах файлов \verb!title.tex!  отдельно для
диссертации и автореферата. Если вам логотип не~нужен, то просто
удалите файл с~логотипом.

\ifsynopsis
Этот абзац появляется только в~автореферате.
Для формирования блоков, которые будут обрабатываться только в~автореферате,
заведена проверка условия \verb!\!\verb!ifsynopsis!.
Значение условия задаётся в~основном файле документа (\verb!synopsis.tex! для
автореферата).
\else
Этот абзац появляется только в~диссертации.
Через проверку условия \verb!\!\verb!ifsynopsis!, задаваемого в~основном файле
документа (\verb!dissertation.tex! для диссертации), можно сделать новую
команду, обеспечивающую появление цитаты в~диссертации, но~не~в~автореферате.
\fi

% {\progress}
% Этот раздел должен быть отдельным структурным элементом по
% ГОСТ, но он, как правило, включается в описание актуальности
% темы. Нужен он отдельным структурынм элемементом или нет ---
% смотрите другие диссертации вашего совета, скорее всего не нужен.

{\aim} данной работы является \ldots

Для~достижения поставленной цели необходимо было решить следующие {\tasks}:
\begin{enumerate}[beginpenalty=10000] % https://tex.stackexchange.com/a/476052/104425
  \item Исследовать, разработать, вычислить и~т.\:д. и~т.\:п.
  \item Исследовать, разработать, вычислить и~т.\:д. и~т.\:п.
  \item Исследовать, разработать, вычислить и~т.\:д. и~т.\:п.
  \item Исследовать, разработать, вычислить и~т.\:д. и~т.\:п.
\end{enumerate}


{\novelty}
\begin{enumerate}[beginpenalty=10000] % https://tex.stackexchange.com/a/476052/104425
  \item Впервые \ldots
  \item Впервые \ldots
  \item Было выполнено оригинальное исследование \ldots
\end{enumerate}

{\influence} \ldots

{\methods} \ldots

{\defpositions}
\begin{enumerate}[beginpenalty=10000] % https://tex.stackexchange.com/a/476052/104425
  \item Первое положение
  \item Второе положение
  \item Третье положение
  \item Четвертое положение
\end{enumerate}
В папке Documents можно ознакомиться с решением совета из Томского~ГУ
(в~файле \verb+Def_positions.pdf+), где обоснованно даются рекомендации
по~формулировкам защищаемых положений.

{\reliability} полученных результатов обеспечивается \ldots \ Результаты находятся в соответствии с результатами, полученными другими авторами.


{\probation}
Основные результаты работы докладывались~на:
перечисление основных конференций, симпозиумов и~т.\:п.

{\contribution} Автор принимал активное участие \ldots

\ifnumequal{\value{bibliosel}}{0}
{%%% Встроенная реализация с загрузкой файла через движок bibtex8. (При желании, внутри можно использовать обычные ссылки, наподобие `\cite{vakbib1,vakbib2}`).
    {\publications} Основные результаты по теме диссертации изложены
    в~XX~печатных изданиях,
    X из которых изданы в журналах, рекомендованных ВАК,
    X "--- в тезисах докладов.
}%
{%%% Реализация пакетом biblatex через движок biber
    \begin{refsection}[bl-author, bl-registered]
        % Это refsection=1.
        % Процитированные здесь работы:
        %  * подсчитываются, для автоматического составления фразы "Основные результаты ..."
        %  * попадают в авторскую библиографию, при usefootcite==0 и стиле `\insertbiblioauthor` или `\insertbiblioauthorgrouped`
        %  * нумеруются там в зависимости от порядка команд `\printbibliography` в этом разделе.
        %  * при использовании `\insertbiblioauthorgrouped`, порядок команд `\printbibliography` в нём должен быть тем же (см. biblio/biblatex.tex)
        %
        % Невидимый библиографический список для подсчёта количества публикаций:
        \printbibliography[heading=nobibheading, section=1, env=countauthorvak,          keyword=biblioauthorvak]%
        \printbibliography[heading=nobibheading, section=1, env=countauthorwos,          keyword=biblioauthorwos]%
        \printbibliography[heading=nobibheading, section=1, env=countauthorscopus,       keyword=biblioauthorscopus]%
        \printbibliography[heading=nobibheading, section=1, env=countauthorconf,         keyword=biblioauthorconf]%
        \printbibliography[heading=nobibheading, section=1, env=countauthorother,        keyword=biblioauthorother]%
        \printbibliography[heading=nobibheading, section=1, env=countregistered,         keyword=biblioregistered]%
        \printbibliography[heading=nobibheading, section=1, env=countauthorpatent,       keyword=biblioauthorpatent]%
        \printbibliography[heading=nobibheading, section=1, env=countauthorprogram,      keyword=biblioauthorprogram]%
        \printbibliography[heading=nobibheading, section=1, env=countauthor,             keyword=biblioauthor]%
        \printbibliography[heading=nobibheading, section=1, env=countauthorvakscopuswos, filter=vakscopuswos]%
        \printbibliography[heading=nobibheading, section=1, env=countauthorscopuswos,    filter=scopuswos]%
        %
        \nocite{*}%
        %
        {\publications} Основные результаты по теме диссертации изложены в~\arabic{citeauthor}~печатных изданиях,
        \arabic{citeauthorvak} из которых изданы в журналах, рекомендованных ВАК\sloppy%
        \ifnum \value{citeauthorscopuswos}>0%
            , \arabic{citeauthorscopuswos} "--- в~периодических научных журналах, индексируемых Web of~Science и Scopus\sloppy%
        \fi%
        \ifnum \value{citeauthorconf}>0%
            , \arabic{citeauthorconf} "--- в~тезисах докладов.
        \else%
            .
        \fi%
        \ifnum \value{citeregistered}=1%
            \ifnum \value{citeauthorpatent}=1%
                Зарегистрирован \arabic{citeauthorpatent} патент.
            \fi%
            \ifnum \value{citeauthorprogram}=1%
                Зарегистрирована \arabic{citeauthorprogram} программа для ЭВМ.
            \fi%
        \fi%
        \ifnum \value{citeregistered}>1%
            Зарегистрированы\ %
            \ifnum \value{citeauthorpatent}>0%
            \formbytotal{citeauthorpatent}{патент}{}{а}{}\sloppy%
            \ifnum \value{citeauthorprogram}=0 . \else \ и~\fi%
            \fi%
            \ifnum \value{citeauthorprogram}>0%
            \formbytotal{citeauthorprogram}{программ}{а}{ы}{} для ЭВМ.
            \fi%
        \fi%
        % К публикациям, в которых излагаются основные научные результаты диссертации на соискание учёной
        % степени, в рецензируемых изданиях приравниваются патенты на изобретения, патенты (свидетельства) на
        % полезную модель, патенты на промышленный образец, патенты на селекционные достижения, свидетельства
        % на программу для электронных вычислительных машин, базу данных, топологию интегральных микросхем,
        % зарегистрированные в установленном порядке.(в ред. Постановления Правительства РФ от 21.04.2016 N 335)
    \end{refsection}%
    \begin{refsection}[bl-author, bl-registered]
        % Это refsection=2.
        % Процитированные здесь работы:
        %  * попадают в авторскую библиографию, при usefootcite==0 и стиле `\insertbiblioauthorimportant`.
        %  * ни на что не влияют в противном случае
        \nocite{vakbib2}%vak
        \nocite{patbib1}%patent
        \nocite{progbib1}%program
        \nocite{bib1}%other
        \nocite{confbib1}%conf
    \end{refsection}%
        %
        % Всё, что вне этих двух refsection, это refsection=0,
        %  * для диссертации - это нормальные ссылки, попадающие в обычную библиографию
        %  * для автореферата:
        %     * при usefootcite==0, ссылка корректно сработает только для источника из `external.bib`. Для своих работ --- напечатает "[0]" (и даже Warning не вылезет).
        %     * при usefootcite==1, ссылка сработает нормально. В авторской библиографии будут только процитированные в refsection=0 работы.
}

При использовании пакета \verb!biblatex! будут подсчитаны все работы, добавленные
в файл \verb!biblio/author.bib!. Для правильного подсчёта работ в~различных
системах цитирования требуется использовать поля:
\begin{itemize}
        \item \texttt{authorvak} если публикация индексирована ВАК,
        \item \texttt{authorscopus} если публикация индексирована Scopus,
        \item \texttt{authorwos} если публикация индексирована Web of Science,
        \item \texttt{authorconf} для докладов конференций,
        \item \texttt{authorpatent} для патентов,
        \item \texttt{authorprogram} для зарегистрированных программ для ЭВМ,
        \item \texttt{authorother} для других публикаций.
\end{itemize}
Для подсчёта используются счётчики:
\begin{itemize}
        \item \texttt{citeauthorvak} для работ, индексируемых ВАК,
        \item \texttt{citeauthorscopus} для работ, индексируемых Scopus,
        \item \texttt{citeauthorwos} для работ, индексируемых Web of Science,
        \item \texttt{citeauthorvakscopuswos} для работ, индексируемых одной из трёх баз,
        \item \texttt{citeauthorscopuswos} для работ, индексируемых Scopus или Web of~Science,
        \item \texttt{citeauthorconf} для докладов на конференциях,
        \item \texttt{citeauthorother} для остальных работ,
        \item \texttt{citeauthorpatent} для патентов,
        \item \texttt{citeauthorprogram} для зарегистрированных программ для ЭВМ,
        \item \texttt{citeauthor} для суммарного количества работ.
\end{itemize}
% Счётчик \texttt{citeexternal} используется для подсчёта процитированных публикаций;
% \texttt{citeregistered} "--- для подсчёта суммарного количества патентов и программ для ЭВМ.

Для добавления в список публикаций автора работ, которые не были процитированы в
автореферате, требуется их~перечислить с использованием команды \verb!\nocite! в
\verb!Synopsis/content.tex!.
