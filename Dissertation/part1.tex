\chapter{Космическое излучение и гамма-астрономия}\label{ch:ch1}
Как было отмеченно во введении, гамма-кванты позволяют детектировать объекты в которых происходит ускорение космических лучей. Это утверждение справедливо по той причине, что высокоэнергичное излучение имеет неравновесную природу, то есть - излучающее тело не находится в термодинамическом равновесии со своим излучением. Равновесное излучение характерно для низкоэнергичных фотонов, а их спектральное распределение определяется температурой тела. Примером подобных объектов могут служить звезды, в частности Солнце, максимум  излучения которого приходится на длину волны фотонов 450 нм.

Излучение в области высоких энергий возникает благодаря тому что в них приходит ускорение заряженных частиц. Далее в этой главе будут рассмотрены механизмы ускорения космических лучей и генерации гамма-квантов.

\section{Ускорение космических лучей}\label{sec:ch1/sec1}
В основе теории ускорения космических лучей лежат механизмы Ферми 1 и 2 рода. Они не требуют наличия специфических структур магнитных и электрических полей, в связи с чем могут объяснить ускорение заряженных частиц практически в любом объекте, наблюдаемом в гамма-квантах высоких энергий.
Механизм первого рода основан на том, что нерегулярное галактическое магнитное поле можно представить как пространство заполненное "магнитными облаками", отражаясь от которых, частица стохастически приобретает энергию, среднее значение которой дается следующим выражением: 
$$\left \langle\frac{\Delta E}{E}\right \rangle = \frac{8}{3}\left(\frac{u}{c}\right )^2,$$
где u - скорость магнитных облаков, движущихся в случайном направлении, c - скорость света.

Механизм второго рода описывает ускорение заряженных частиц на ударных волнах, которые нередко присутствуют вблизи объектов с характеризующихся высоким темпом "утекания" вещества. Магнитное поле, связанное с ударной волной рассеивает частицы, в результате чего они могут многократно пересекать фронт. Заряженная частица пересекшая фронт и вернувшаяся обратно получает приращение энергии:
$$\left \langle\frac{\Delta E}{E}\right \rangle = \frac{4}{3}\frac{u}{c},$$
где u - скорость ударной волны.

В последние годы также все больше внимания привлекают исследования ускорения частиц на пересоединении линий магнитного поля. Благодаря выросшим возможностям численного МГД-моделирования, исследователям удалось показать, что энергия магнитного поля может эффективно переходить в энергию заряженных частиц.

Описанные механизмы, по всей видимости, действуют в большинстве предполагаемых ускорителей космических лучей, однако, поскольку частицы высоких энергий зарегистрированные у Земли не могут быть достоверно ассоциированны с каким-либо источником, количественно проверить где и в какой пропорции задействованны те или иные механизмы достаточно сложно. Тем не менее, энергетические спектры, которые могут быть полученны посредством данных механизмов имеют степенной наклон и могут описывать экспериментально наблюдаемый спектр космических лучей.

Помимо наклона спектра КЛ, важным также представляется объяснить происхождение его особенностей. Наиболее заметными и привлекающими внимание исследователей являются изломы при энергиях $3*10^{15}$эВ и $10^{18.6}$эВ  Первый излом - так называемое "колено" предположительно связан с тем, что при данной энергии Ларморовский радиус протонов становится сопоставим с размерами галактики и частицы более высоких энергий свободно ее покидают. Известно что структура колена сложная, что может быть связанно с тем, что для ядер космических лучей с б\'ольшим Z, Ларморовский радиус становится сопоставим с размером галактики при б\'ольших энергиях и, как следствие, переход от галактических к внегалактическим КЛ не является резким. Таким образом, предполагается что частицы с $E < 3*10^{15}$эВ имеют галактическое происхождение, а более высокоэнергичные частицы могут иметь внегалактическую природу. 

Излом при энергии $10^{18.6}$эВ - лодыжка, менее изучен в силу трудностей накопления статистики. Однако, одим из предположений его возникновения является механизм фотодиссоциации ядер ультравысоких энергий в области около ускорителя\cite{Unger_2015}.

Несмотря на то что экспериментально источники КЛ до сих пор не открыты, вспышки сверхновых уже долгое время считаются основным кандидатом на эту роль. Данная теория была предложена в 1963 году Гинзбургом и Сыроватским. При вспышке выделяется энергия $\sim10^{51}$эрг. Наблюдаемая частота вспышек находится в диапазоне 30-40 лет. Таким образом, мощность, генерируемая при взрывах сверхновых составит $\sim10^{42}$эрг/с. При этом, плотность энергии космических лучей составляет $\sim10^{-12}\text{эрг/см}^{-3}$. Учитывая, что радиус галактики 15кпк, а толщина 300пк, для поддержания наблюдаемой плотности энергии КЛ требуется  $\sim10^{41}$эрг/с. Таким образом, вспышки сверхновых являются наиболее удовлетворительным кандидатом на роль ускорителей галактических КЛ.

Однако, экспериментальная проверка данной теории достаточно сложна, поскольку для выполнения этой задачи требуется доказать присутствие в остатках сверхновых частиц, ускоренных до энергий $\sim3*10^{15}$эВ. Обнаружить их напрямую, как уже было отмеченно в начале главы, нельзя. Однако это может быть сделанно посредством гамма-астрономии.


\section{Гамма-астрономия}\label{sec:ch1/sec2}
Косвенным признаком того что в объекте происходит ускорение заряженных частиц является наличие гамма-излучения от этого источника. Существует ряд механизмов генерации гамма-квантов, которые могут быть разделены по признаку родительских частиц - адронные и лептонные.
\subsection{Лептонные механизмы}\label{sec:ch1/sec2/sec1}
Лептонные механизмы рождения гамма-квантов включают процессы, в которых электроны или позитроны участвуют в генерации высокоэнергетических гамма-квантов. Основными из них являются:
\begin{enumerate}[beginpenalty=10000] 
	\item Обратное комптоновское рассеяние - процесс при котором низкоэнергичный гамма-квант может рассеяться на высокоэнергичном электроне, и получить приращение энергии за счет уменьшения энергии электрона. Данный процесс эффективно работает когда энергия гамма-кванта $E_{\gamma}$ существенно меньше энергии электрона $E_{e^-}$. В случае $E_{\gamma} \approx E_{e^-}$ происходит Томсоновское рассеяние, не сопровождающееся передачей энергии. Если родительские электроны имеют энергетический спектр с наклоном $-p$, тогда формируемый в результате обратного комптоновского рассеяния энергетический спектр гамма-квантов дается следующим выражением: $\phi \sim E_{\gamma}^{-(p+1)}$
	\item Синхротронное излучение: заряженные частицы в магнитном поле движутся под действием силы Лоренца по спиралевидным траекториям. Следовательно, частицы обладают центростремительным ускорением и излучают электромагнитные волны. В нерелятивистском случае излучение имеет симметричную диаграмму направленности, дискретный спектр излучения, связанный с частотой вращения частицы в магнитном поле и называется циклотронным излучением. В релятивистском случае излучение сконцентрированно вдоль вектора мгновенной скорости частицы и линейно поляризованно. Такой режим называется синхротронным и формирует спектр гамма-квантов  $\phi \sim E_{\gamma}^{-(p+1)/2}$. Интенсивность излучения пропорциональна $1/m^4$, где m - масса покоя заряженной частицы. Таким образом электроны и позитроны являются существенно более эффективными генераторами синхротронного излучения. 
	\item Тормозное излучение: при прохождении заряженной частицы вблизи атомных ядер ее траектория искривляется, в результате чего возникает электромагнитное излучение. В релятивистском случае оно, так же как и для синхротронного излучения, сконцентрированно вдоль вектора мгновенной скорости электрона. Интенсивность пропорциональна $1/m^2$, а результирующий спектр гамма-квантов  $\phi \sim E_{\gamma}^{-p}$.
\end{enumerate}
\subsection{Адронный механизм}\label{sec:ch1/sec2/sec1}
В основе данного механизма лежит распад нейтрального $\pi$-мезона ($\pi^0$). $\pi$-мезоны бывают заряженные ($\pi^{\pm}$) и нейтральные ($\pi^0$) и рождаются при взаимодействии высокоэнергичных адронов с частицами окружающей среды. Среднее время жизни $\pi^0$ составляет $8.2*10^{-17}c$, после чего он распадается на 2 гамма-кванта: $\pi^0\rightarrow2\gamma$. Распад заряженного пи-мезона имеет следующий вид: $\pi^{\pm}\rightarrow\mu^{\pm} + \nu_{\mu}(\nu_{\={\mu}})$. Таким образом, адронный механизм генерации гамма-квантов характеризуется и нейтринной составляющей, обнаружение которой также может позволить определить источники ускорения космических лучей. Результирующий спектр гамма-квантов в данном случае будет иметь тот же наклон что и энергетический спектр родительских  $\pi^0$-мезонов.

Адронный и лептонный механизмы позволяют получать энергетические спектры описывающие экспериментальные данные наблюдения источников гамма-квантов сверхвысоких энергий. Возможность определить какой из них преобладает в исследуемом источнике возникает при исследовании энергетической области выше 10 ТэВ. Лептонный спектр имеет тенденцию к более резкому обрезанию в области до 10 ТэВ, в то время как адронный является более пологим и продолжается до более высоких энергий. Таким образом, для определения природы высокоэнергичного излучения требуется исследование энергетической области выше 1 ТэВ. Обнаружение гамма-квантов с энергиями порядка 100ТэВ будет косвенно свидетельствовать об адронной природе гамма-излучения и, как следствие возможности ускорения космических лучей в данном источнике до области колена.

\subsection{Источники гамма-излучения}
С момента детектирования гамма-излучения очень высоких энергий от Крабовидной туманности в 1988 году были открыты десятки других источников. Как ни странно, среди них есть не только остатки сверхновых, но также пульсарные туманности (PWN) и звездные кластеры.

\subsubsection{Пульсарные туманности}
Наиболее распространенным типом объектов, наблюдаемых в гамма-излучении очень высоких энергий являются пульсарные туманности. Это класс объектов содержащих оболочку, образуемую потоком ультрарелятивистских электронов и позитронов исходящих от пульсара - быстровращающейся нейтронной звезды с сильным магнитным полем находящейся внутри оболочки. 

Сейчас экспериментальные исследования как в области гамма-астрономии, так и в области физики космических лучей можно разделить на прямы исследования и косвенные. Прямые измерения проводятся со спутниковых обсерваторий, поскольку атмосфера земли для гамма-квантов непрозрачна. С начала 90-х годов на орбиту были запущены такие спутники как CRGO, INTEGRAL, Swift, AGILE и Fermi, наблюдения которых дали огромное количество информации о источниках гамма-излучения как в нашей галактике, так и за ее пределами. 

Прямые измерения позволяют с высокой точностью восстанавливать параметры регистрируемых частиц и получать энергетические спектры источников излучения в высоком разрешении. Однако, в связи с тем, что потоки гамма-квантов имеют степенную зависимость от энергии частиц, а эффективные площади орбитальных обсерваторий не превышают 1$м^2$, набрать существенную статистику в энергетической области выше 1 ТэВ становится очень сложно. По этой причине, спутниковые гамма-обсерватории преимущественно работают в энергетической области МэВ-ГэВ. 

При более высоких энергиях применяется косвенный метод регистрации гамма-квантов, в котором атмосфера Земли выступает в роли калориметра, попадая в который первичная частица взаимодействует с атомами атмосферы, рождая поток вторичных космических лучей. Данное явление называется Широким Атмосферным Ливнем (ШАЛ) и широко используется для наземного детектирования космических лучей и гамма-квантов. При данном подходе удается покрыть большую площадь и как следствие - иметь большую статистику в энергетической области выше 1 ТэВ. Однако возникает проблема выделения гамма-квантов из адронного фона, поскольку ШАЛы от адронов имеют сходную сигнатуру с гамма-квантами. Данный вопрос будет рассмотрен в главе 2. 



Мы можем сделать \textbf{жирный текст} и \textit{курсив}.

\section{Ссылки}\label{sec:ch1/sec2}

Сошлёмся на библиографию.
Одна ссылка: \cite[с.~54]{Sokolov}\cite[с.~36]{Gaidaenko}.
Две ссылки: \cite{Sokolov,Gaidaenko}.
Ссылка на собственные работы: \cite{vakbib1, confbib2}.
Много ссылок: %\cite[с.~54]{Lermontov,Management,Borozda} % такой «фокус»
%вызывает biblatex warning относительно опции sortcites, потому что неясно, к
%какому источнику относится уточнение о страницах, а bibtex об этой проблеме
%даже не предупреждает
\cite{Lermontov, Management, Borozda, Marketing, Constitution, FamilyCode,
    Gost.7.0.53, Razumovski, Lagkueva, Pokrovski, Methodology, Berestova,
    Kriger}%
\ifnumequal{\value{bibliosel}}{0}{% Примеры для bibtex8
    \cite{Sirotko, Lukina, Encyclopedia, Nasirova}%
}{% Примеры для biblatex через движок biber
    \cite{Sirotko2, Lukina2, Encyclopedia2, Nasirova2}%
}%
.
И~ещё немного ссылок:~\cite{Article,Book,Booklet,Conference,Inbook,Incollection,Manual,Mastersthesis,
    Misc,Phdthesis,Proceedings,Techreport,Unpublished}
% Следует обратить внимание, что пробел после запятой внутри \cite{}
% обрабатывается ожидаемо, а пробел перед запятой, может вызывать проблемы при
% обработке ссылок.
\cite{medvedev2006jelektronnye, CEAT:CEAT581, doi:10.1080/01932691.2010.513279,
    Gosele1999161,Li2007StressAnalysis, Shoji199895, test:eisner-sample,
    test:eisner-sample-shorted, AB_patent_Pomerantz_1968, iofis_patent1960}%
\ifnumequal{\value{bibliosel}}{0}{% Примеры для bibtex8
}{% Примеры для biblatex через движок biber
    \cite{patent2h, patent3h, patent2}%
}%
.

\ifnumequal{\value{bibliosel}}{0}{% Примеры для bibtex8
Попытка реализовать несколько ссылок на конкретные страницы
для \texttt{bibtex} реализации библиографии:
[\citenum{Sokolov}, с.~54; \citenum{Gaidaenko}, с.~36].
}{% Примеры для biblatex через движок biber
Несколько источников (мультицитата):
% Тут специально написано по-разному тире, для демонстрации, что
% применение специальных тире в настоящий момент в biblatex приводит к непоказу
% "с.".
\cites[vii--x, 5, 7]{Sokolov}[v"--~x, 25, 526]{Gaidaenko}[vii--x, 5, 7]{Techreport},
работает только в \texttt{biblatex} реализации библиографии.
}%

Ссылки на собственные работы:~\cite{vakbib1, confbib1}.

Сошлёмся на приложения: Приложение~\cref{app:A}, Приложение~\cref{app:B2}.

Сошлёмся на формулу: формула~\cref{eq:equation1}.

Сошлёмся на изображение: рисунок~\cref{fig:knuth}.

Стандартной практикой является добавление к ссылкам префикса, характеризующего тип элемента.
Это не является строгим требованием, но~позволяет лучше ориентироваться в документах большого размера.
Например, для ссылок на~рисунки используется префикс \textit{fig},
для ссылки на~таблицу "--- \textit{tab}.

В таблице \cref{tab:tab_pref} приложения~\cref{app:B4} приведён список рекомендуемых
к использованию стандартных префиксов.

В некоторых ситуациях возникает необходимость отойти от требований ГОСТ по оформлению ссылок на
литературу.
В таком случае можно воспользоваться дополнительными опциями пакета \verb+biblatex+.

Например, в ссылке на книгу~\cite{sobenin_kdv} использование опции \verb+maxnames=4+ позволяет
вывести имена всех четырёх авторов.
По ГОСТ имена последних трёх авторов опускаются.

Кроме того, часто возникают проблемы с транслитерованными инициалами. Некоторые буквы русского
алфавита по правилам транслитерации записываются двумя буквами латинского алфавита (ю-yu, ё-yo и
т.д.).
Такие инициалы \verb+biblatex+ будет сокращать до одной буквы, что неверно.
Поправить его работу можно использовав опцию \verb+giveninits=false+.
Пример использования этой опции можно видеть в ссылке~\cite{initials}.

\section{Формулы}\label{sec:ch1/sec3}

Благодаря пакету \textit{icomma}, \LaTeX~одинаково хорошо воспринимает
в~качестве десятичного разделителя и запятую (\(3,1415\)), и точку (\(3.1415\)).

\subsection{Ненумерованные одиночные формулы}\label{subsec:ch1/sec3/sub1}

Вот так может выглядеть формула, которую необходимо вставить в~строку
по~тексту: \(x \approx \sin x\) при \(x \to 0\).

А вот так выглядит ненумерованная отдельностоящая формула c подстрочными
и надстрочными индексами:
\[
    (x_1+x_2)^2 = x_1^2 + 2 x_1 x_2 + x_2^2
\]

Формула с неопределенным интегралом:
\[
    \int f(\alpha+x)=\sum\beta
\]

При использовании дробей формулы могут получаться очень высокие:
\[
    \frac{1}{\sqrt{2}+
        \displaystyle\frac{1}{\sqrt{2}+
            \displaystyle\frac{1}{\sqrt{2}+\cdots}}}
\]

В формулах можно использовать греческие буквы:
%Все \original... команды заранее, ради этого примера, определены в Dissertation\userstyles.tex
\[
    \alpha\beta\gamma\delta\originalepsilon\epsilon\zeta\eta\theta%
    \vartheta\iota\kappa\varkappa\lambda\mu\nu\xi\pi\varpi\rho\varrho%
    \sigma\varsigma\tau\upsilon\originalphi\phi\chi\psi\omega\Gamma\Delta%
    \Theta\Lambda\Xi\Pi\Sigma\Upsilon\Phi\Psi\Omega
\]
\[%https://texfaq.org/FAQ-boldgreek
    \boldsymbol{\alpha\beta\gamma\delta\originalepsilon\epsilon\zeta\eta%
        \theta\vartheta\iota\kappa\varkappa\lambda\mu\nu\xi\pi\varpi\rho%
        \varrho\sigma\varsigma\tau\upsilon\originalphi\phi\chi\psi\omega\Gamma%
        \Delta\Theta\Lambda\Xi\Pi\Sigma\Upsilon\Phi\Psi\Omega}
\]

Для добавления формул можно использовать пары \verb+$+\dots\verb+$+ и \verb+$$+\dots\verb+$$+,
но~они считаются устаревшими.
Лучше использовать их функциональные аналоги \verb+\(+\dots\verb+\)+ и \verb+\[+\dots\verb+\]+.

\subsection{Ненумерованные многострочные формулы}\label{subsec:ch1/sec3/sub2}

Вот так можно написать две формулы, не нумеруя их, чтобы знаки <<равно>> были
строго друг под другом:
\begin{align}
    f_W & =  \min \left( 1, \max \left( 0, \frac{W_{soil} / W_{max}}{W_{crit}} \right)  \right), \nonumber \\
    f_T & =  \min \left( 1, \max \left( 0, \frac{T_s / T_{melt}}{T_{crit}} \right)  \right), \nonumber
\end{align}

Выровнять систему ещё и по переменной \( x \) можно, используя окружение
\verb|alignedat| из пакета \verb|amsmath|. Вот так:
\[
|x| = \left\{
\begin{alignedat}{2}
    &&x, \quad &\text{eсли } x\geqslant 0 \\
    &-&x, \quad & \text{eсли } x<0
\end{alignedat}
\right.
\]
Здесь первый амперсанд (в исходном \LaTeX\ описании формулы) означает
выравнивание по~левому краю, второй "--- по~\( x \), а~третий "--- по~слову
<<если>>. Команда \verb|\quad| делает большой горизонтальный пробел.

Ещё вариант:
\[
    |x|=
    \begin{cases}
        \phantom{-}x, \text{если } x \geqslant 0 \\
        -x, \text{если } x<0
    \end{cases}
\]

Кроме того, для  нумерованных формул \verb|alignedat| делает вертикальное
выравнивание номера формулы по центру формулы. Например, выравнивание
компонент вектора:
\begin{equation}
    \label{eq:2p3}
    \begin{alignedat}{2}
        {\mathbf{N}}_{o1n}^{(j)} = \,{\sin} \phi\,n\!\left(n+1\right)
        {\sin}\theta\,
        \pi_n\!\left({\cos} \theta\right)
        \frac{
        z_n^{(j)}\!\left( \rho \right)
        }{\rho}\,
        &{\boldsymbol{\hat{\mathrm e}}}_{r}\,+   \\
        +\,
        {\sin} \phi\,
        \tau_n\!\left({\cos} \theta\right)
        \frac{
        \left[\rho z_n^{(j)}\!\left( \rho \right)\right]^{\prime}
        }{\rho}\,
        &{\boldsymbol{\hat{\mathrm e}}}_{\theta}\,+   \\
        +\,
        {\cos} \phi\,
        \pi_n\!\left({\cos} \theta\right)
        \frac{
        \left[\rho z_n^{(j)}\!\left( \rho \right)\right]^{\prime}
        }{\rho}\,
        &{\boldsymbol{\hat{\mathrm e}}}_{\phi}\:.
    \end{alignedat}
\end{equation}

Ещё об отступах. Иногда для лучшей <<читаемости>> формул полезно
немного исправить стандартные интервалы \LaTeX\ с учётом логической
структуры самой формулы. Например в формуле~\cref{eq:2p3} добавлен
небольшой отступ \verb+\,+ между основными сомножителями, ниже
результат применения всех вариантов отступа:
\begin{align*}
    \backslash!             & \quad f(x) = x^2\! +3x\! +2         \\
    \mbox{по-умолчанию}     & \quad f(x) = x^2+3x+2               \\
    \backslash,             & \quad f(x) = x^2\, +3x\, +2         \\
    \backslash{:}           & \quad f(x) = x^2\: +3x\: +2         \\
    \backslash;             & \quad f(x) = x^2\; +3x\; +2         \\
    \backslash \mbox{space} & \quad f(x) = x^2\ +3x\ +2           \\
    \backslash \mbox{quad}  & \quad f(x) = x^2\quad +3x\quad +2   \\
    \backslash \mbox{qquad} & \quad f(x) = x^2\qquad +3x\qquad +2
\end{align*}

Можно использовать разные математические алфавиты:
\begin{align}
    \mathcal{ABCDEFGHIJKLMNOPQRSTUVWXYZ} \nonumber  \\
    \mathfrak{ABCDEFGHIJKLMNOPQRSTUVWXYZ} \nonumber \\
    \mathbb{ABCDEFGHIJKLMNOPQRSTUVWXYZ} \nonumber
\end{align}

Посмотрим на систему уравнений на примере аттрактора Лоренца:

\[
\left\{
\begin{array}{rl}
    \dot x = & \sigma (y-x)  \\
    \dot y = & x (r - z) - y \\
    \dot z = & xy - bz
\end{array}
\right.
\]

А для вёрстки матриц удобно использовать многоточия:
\[
    \left(
        \begin{array}{ccc}
            a_{11} & \ldots & a_{1n} \\
            \vdots & \ddots & \vdots \\
            a_{n1} & \ldots & a_{nn} \\
        \end{array}
    \right)
\]

\subsection{Нумерованные формулы}\label{subsec:ch1/sec3/sub3}

А вот так пишется нумерованная формула:
\begin{equation}
    \label{eq:equation1}
    e = \lim_{n \to \infty} \left( 1+\frac{1}{n} \right) ^n
\end{equation}

Нумерованных формул может быть несколько:
\begin{equation}
    \label{eq:equation2}
    \lim_{n \to \infty} \sum_{k=1}^n \frac{1}{k^2} = \frac{\pi^2}{6}
\end{equation}

Впоследствии на формулы~\cref{eq:equation1, eq:equation2} можно ссылаться.

Сделать так, чтобы номер формулы стоял напротив средней строки, можно,
используя окружение \verb|multlined| (пакет \verb|mathtools|) вместо
\verb|multline| внутри окружения \verb|equation|. Вот так:
\begin{equation} % \tag{S} % tag - вписывает свой текст
    \label{eq:equation3}
    \begin{multlined}
        1+ 2+3+4+5+6+7+\dots + \\
        + 50+51+52+53+54+55+56+57 + \dots + \\
        + 96+97+98+99+100=5050
    \end{multlined}
\end{equation}

Уравнения~\cref{eq:subeq_1,eq:subeq_2} демонстрируют возможности
окружения \verb|\subequations|.
\begin{subequations}
    \label{eq:subeq_1}
    \begin{gather}
        y = x^2 + 1 \label{eq:subeq_1-1} \\
        y = 2 x^2 - x + 1 \label{eq:subeq_1-2}
    \end{gather}
\end{subequations}
Ссылки на отдельные уравнения~\cref{eq:subeq_1-1,eq:subeq_1-2,eq:subeq_2-1}.
\begin{subequations}
    \label{eq:subeq_2}
    \begin{align}
        y & = x^3 + x^2 + x + 1 \label{eq:subeq_2-1} \\
        y & = x^2
    \end{align}
\end{subequations}

\subsection{Форматирование чисел и размерностей величин}\label{sec:units}

Числа форматируются при помощи команды \verb|\num|:
\num{5,3};
\num{2,3e8};
\num{12345,67890};
\num{2,6 d4};
\num{1+-2i};
\num{.3e45};
\num[exponent-base=2]{5 e64};
\num[exponent-base=2,exponent-to-prefix]{5 e64};
\num{1.654 x 2.34 x 3.430}
\num{1 2 x 3 / 4}.
Для написания последовательности чисел можно использовать команды \verb|\numlist| и \verb|\numrange|:
\numlist{10;30;50;70}; \numrange{10}{30}.
Значения углов можно форматировать при помощи команды \verb|\ang|:
\ang{2.67};
\ang{30,3};
\ang{-1;;};
\ang{;-2;};
\ang{;;-3};
\ang{300;10;1}.

Обратите внимание, что ГОСТ запрещает использование знака <<->> для обозначения отрицательных чисел
за исключением формул, таблиц и~рисунков.
Вместо него следует использовать слово <<минус>>.

Размерности можно записывать при помощи команд \verb|\si| и \verb|\SI|:
\si{\farad\squared\lumen\candela};
\si{\joule\per\mole\per\kelvin};
\si[per-mode = symbol-or-fraction]{\joule\per\mole\per\kelvin};
\si{\metre\per\second\squared};
\SI{0.10(5)}{\neper};
\SI{1.2-3i e5}{\joule\per\mole\per\kelvin};
\SIlist{1;2;3;4}{\tesla};
\SIrange{50}{100}{\volt}.
Список единиц измерений приведён в таблицах~\cref{tab:unit:base,
    tab:unit:derived,tab:unit:accepted,tab:unit:physical,tab:unit:other}.
Приставки единиц приведены в~таблице~\cref{tab:unit:prefix}.

С дополнительными опциями форматирования можно ознакомиться в~описании пакета \texttt{siunitx};
изменить или добавить единицы измерений можно в~файле \texttt{siunitx.cfg}.

\begin{table}
    \centering
    \captionsetup{justification=centering} % выравнивание подписи по-центру
    \caption{Основные величины СИ}\label{tab:unit:base}
    \begin{tabular}{llc}
        \toprule
        Название  & Команда                 & Символ         \\
        \midrule
        Ампер     & \verb|\ampere| & \si{\ampere}   \\
        Кандела   & \verb|\candela| & \si{\candela}  \\
        Кельвин   & \verb|\kelvin| & \si{\kelvin}   \\
        Килограмм & \verb|\kilogram| & \si{\kilogram} \\
        Метр      & \verb|\metre| & \si{\metre}    \\
        Моль      & \verb|\mole| & \si{\mole}     \\
        Секунда   & \verb|\second| & \si{\second}   \\
        \bottomrule
    \end{tabular}
\end{table}

\begin{table}
    \small
    \centering
    \begin{threeparttable}% выравнивание подписи по границам таблицы
        \caption{Производные единицы СИ}\label{tab:unit:derived}
        \begin{tabular}{llc|llc}
            \toprule
            Название       & Команда                 & Символ              & Название & Команда & Символ \\
            \midrule
            Беккерель      & \verb|\becquerel| & \si{\becquerel}     &
            Ньютон         & \verb|\newton| & \si{\newton}                                      \\
            Градус Цельсия & \verb|\degreeCelsius| & \si{\degreeCelsius} &
            Ом             & \verb|\ohm| & \si{\ohm}                                         \\
            Кулон          & \verb|\coulomb| & \si{\coulomb}       &
            Паскаль        & \verb|\pascal| & \si{\pascal}                                      \\
            Фарад          & \verb|\farad| & \si{\farad}         &
            Радиан         & \verb|\radian| & \si{\radian}                                      \\
            Грей           & \verb|\gray| & \si{\gray}          &
            Сименс         & \verb|\siemens| & \si{\siemens}                                     \\
            Герц           & \verb|\hertz| & \si{\hertz}         &
            Зиверт         & \verb|\sievert| & \si{\sievert}                                     \\
            Генри          & \verb|\henry| & \si{\henry}         &
            Стерадиан      & \verb|\steradian| & \si{\steradian}                                   \\
            Джоуль         & \verb|\joule| & \si{\joule}         &
            Тесла          & \verb|\tesla| & \si{\tesla}                                       \\
            Катал          & \verb|\katal| & \si{\katal}         &
            Вольт          & \verb|\volt| & \si{\volt}                                        \\
            Люмен          & \verb|\lumen| & \si{\lumen}         &
            Ватт           & \verb|\watt| & \si{\watt}                                        \\
            Люкс           & \verb|\lux| & \si{\lux}           &
            Вебер          & \verb|\weber| & \si{\weber}                                       \\
            \bottomrule
        \end{tabular}
    \end{threeparttable}
\end{table}

\begin{table}
    \centering
    \begin{threeparttable}% выравнивание подписи по границам таблицы
        \caption{Внесистемные единицы}\label{tab:unit:accepted}

        \begin{tabular}{llc}
            \toprule
            Название        & Команда                 & Символ          \\
            \midrule
            День            & \verb|\day| & \si{\day}       \\
            Градус          & \verb|\degree| & \si{\degree}    \\
            Гектар          & \verb|\hectare| & \si{\hectare}   \\
            Час             & \verb|\hour| & \si{\hour}      \\
            Литр            & \verb|\litre| & \si{\litre}     \\
            Угловая минута  & \verb|\arcminute| & \si{\arcminute} \\
            Угловая секунда & \verb|\arcsecond| & \si{\arcsecond} \\ %
            Минута          & \verb|\minute| & \si{\minute}    \\
            Тонна           & \verb|\tonne| & \si{\tonne}     \\
            \bottomrule
        \end{tabular}
    \end{threeparttable}
\end{table}

\begin{table}
    \centering
    \captionsetup{justification=centering}
    \caption{Внесистемные единицы, получаемые из эксперимента}\label{tab:unit:physical}
    \begin{tabular}{llc}
        \toprule
        Название                & Команда                 & Символ                 \\
        \midrule
        Астрономическая единица & \verb|\astronomicalunit| & \si{\astronomicalunit} \\
        Атомная единица массы   & \verb|\atomicmassunit| & \si{\atomicmassunit}   \\
        Боровский радиус        & \verb|\bohr| & \si{\bohr}             \\
        Скорость света          & \verb|\clight| & \si{\clight}           \\
        Дальтон                 & \verb|\dalton| & \si{\dalton}           \\
        Масса электрона         & \verb|\electronmass| & \si{\electronmass}     \\
        Электрон Вольт          & \verb|\electronvolt| & \si{\electronvolt}     \\
        Элементарный заряд      & \verb|\elementarycharge| & \si{\elementarycharge} \\
        Энергия Хартри          & \verb|\hartree| & \si{\hartree}          \\
        Постоянная Планка       & \verb|\planckbar| & \si{\planckbar}        \\
        \bottomrule
    \end{tabular}
\end{table}

\begin{table}
    \centering
    \begin{threeparttable}% выравнивание подписи по границам таблицы
        \caption{Другие внесистемные единицы}\label{tab:unit:other}
        \begin{tabular}{llc}
            \toprule
            Название                  & Команда                 & Символ             \\
            \midrule
            Ангстрем                  & \verb|\angstrom| & \si{\angstrom}     \\
            Бар                       & \verb|\bar| & \si{\bar}          \\
            Барн                      & \verb|\barn| & \si{\barn}         \\
            Бел                       & \verb|\bel| & \si{\bel}          \\
            Децибел                   & \verb|\decibel| & \si{\decibel}      \\
            Узел                      & \verb|\knot| & \si{\knot}         \\
            Миллиметр ртутного столба & \verb|\mmHg| & \si{\mmHg}         \\
            Морская миля              & \verb|\nauticalmile| & \si{\nauticalmile} \\
            Непер                     & \verb|\neper| & \si{\neper}        \\
            \bottomrule
        \end{tabular}
    \end{threeparttable}
\end{table}

\begin{table}
    \small
    \centering
    \begin{threeparttable}% выравнивание подписи по границам таблицы
        \caption{Приставки СИ}\label{tab:unit:prefix}
        \begin{tabular}{llcc|llcc}
            \toprule
            Приставка & Команда                  & Символ      & Степень &
            Приставка & Команда                  & Символ      & Степень   \\
            \midrule
            Иокто     & \verb|\yocto|  & \si{\yocto} & -24     &
            Дека      & \verb|\deca|  & \si{\deca}  & 1         \\
            Зепто     & \verb|\zepto|  & \si{\zepto} & -21     &
            Гекто     & \verb|\hecto|  & \si{\hecto} & 2         \\
            Атто      & \verb|\atto|  & \si{\atto}  & -18     &
            Кило      & \verb|\kilo|  & \si{\kilo}  & 3         \\
            Фемто     & \verb|\femto|  & \si{\femto} & -15     &
            Мега      & \verb|\mega|  & \si{\mega}  & 6         \\
            Пико      & \verb|\pico|  & \si{\pico}  & -12     &
            Гига      & \verb|\giga|  & \si{\giga}  & 9         \\
            Нано      & \verb|\nano|  & \si{\nano}  & -9      &
            Терра     & \verb|\tera|  & \si{\tera}  & 12        \\
            Микро     & \verb|\micro|  & \si{\micro} & -6      &
            Пета      & \verb|\peta|  & \si{\peta}  & 15        \\
            Милли     & \verb|\milli|  & \si{\milli} & -3      &
            Екса      & \verb|\exa|  & \si{\exa}   & 18        \\
            Санти     & \verb|\centi|  & \si{\centi} & -2      &
            Зетта     & \verb|\zetta|  & \si{\zetta} & 21        \\
            Деци      & \verb|\deci| & \si{\deci}  & -1      &
            Иотта     & \verb|\yotta| & \si{\yotta} & 24        \\
            \bottomrule
        \end{tabular}
    \end{threeparttable}
\end{table}

\subsection{Заголовки с формулами: \texorpdfstring{\(a^2 + b^2 = c^2\)}{%
        a\texttwosuperior\ + b\texttwosuperior\ = c\texttwosuperior},
    \texorpdfstring{\(\left\vert\textrm{{Im}}\Sigma\left(
            \protect\varepsilon\right)\right\vert\approx const\)}{|ImΣ (ε)| ≈ const},
    \texorpdfstring{\(\sigma_{xx}^{(1)}\)}{σ\_\{xx\}\textasciicircum\{(1)\}}
}\label{subsec:with_math}

Пакет \texttt{hyperref} берёт текст для закладок в pdf-файле из~аргументов
команд типа \verb|\section|, которые могут содержать математические формулы,
а~также изменения цвета текста или шрифта, которые не отображаются в~закладках.
Чтобы использование формул в заголовках не вызывало в~логе компиляции появление
предупреждений типа <<\texttt{Token not allowed in~a~PDF string
    (Unicode):(hyperref) removing...}>>, следует использовать конструкцию
\verb|\texorpdfstring{}{}|, где в~первых фигурных скобках указывается
формула, а~во~вторых "--- запись формулы для закладок.

\section{Рецензирование текста}\label{sec:markup}

В шаблоне для диссертации и автореферата заданы команды рецензирования.
Они видны при компиляции шаблона в режиме черновика или при установке
соответствующей настройки (\verb+showmarkup+) в~файле \verb+common/setup.tex+.

Команда \verb+\todo+ отмечает текст красным цветом.
\todo{Например, так.}

Команда \verb+\note+ позволяет выбрать цвет текста.
\note{Чёрный, } \note[red]{красный, } \note[green]{зелёный, }
\note[blue]{синий.} \note[orange]{Обратите внимание на ширину и расстановку
    формирующихся пробелов, в~результате приведённой записи (зависит также
    от~применяемого компилятора).}

Окружение \verb+commentbox+ также позволяет выбрать цвет.

\begin{commentbox}[red]
    Красный текст.

    Несколько параграфов красного текста.
\end{commentbox}

\begin{commentbox}[blue]
    Синяя формула.

    \begin{equation}
        \alpha + \beta = \gamma
    \end{equation}
\end{commentbox}

\verb+commentbox+ позволяет закомментировать участок кода в~режиме чистовика.
Чтобы убрать кусок кода для всех режимов, можно использовать окружение
\verb+comment+.

\begin{comment}
Этот текст всегда скрыт.
\end{comment}

\section{Работа со списком сокращений и~условных обозначений}\label{sec:acronyms}

С помощью пакета \texttt{nomencl} можно создавать удобный сортированный список
сокращений и условных обозначений во время написания текста. Вызов
\verb+\nomenclature+ добавляет нужный символ или сокращение с~описанием
в~список, который затем печатается вызовом \verb+\printnomenclature+
в~соответствующем разделе.
Для того, чтобы эти операции прошли, потребуется дополнительный вызов
\verb+makeindex -s nomencl.ist -o %.nls %.nlo+ в~командной строке, где вместо
\verb+%+ следует подставить имя главного файла проекта (\verb+dissertation+
для этого шаблона).
Затем потребуется один или два дополнительных вызова компилятора проекта.
\begin{equation}
    \omega = c k,
\end{equation}
где \( \omega \) "--- частота света, \( c \) "--- скорость света, \( k \) "---
модуль волнового вектора.
\nomenclature{\(\omega\)}{частота света\nomrefeq}
\nomenclature{\(c\)}{скорость света\nomrefpage}
\nomenclature{\(k\)}{модуль волнового вектора\nomrefeqpage}
Использование
\begin{verbatim}
\nomenclature{\(\omega\)}{частота света\nomrefeq}
\nomenclature{\(c\)}{скорость света\nomrefpage}
\nomenclature{\(k\)}{модуль волнового вектора\nomrefeqpage}
\end{verbatim}
после уравнения добавит в список условных обозначений три записи.
Ссылки \verb+\nomrefeq+ на последнее уравнение, \verb+\nomrefpage+ "--- на
страницу, \verb+\nomrefeqpage+ "--- сразу на~последнее уравнение и~на~страницу,
можно опускать и~не~использовать.

Группировкой и сортировкой пунктов в списке можно управлять с~помощью указания
дополнительных аргументов к команде \verb+nomenclature+.
Например, при вызове
\begin{verbatim}
\nomenclature[03]{\( \hbar \)}{постоянная Планка}
\nomenclature[01]{\( G \)}{гравитационная постоянная}
\end{verbatim}
\( G \) будет стоять в списке выше, чем \( \hbar \).
Для корректных вертикальных отступов между строками в описании лучше
не~использовать многострочные формулы в~списке обозначений.

\nomenclature{%
    \( \begin{rcases}
        a_n \\
        b_n
    \end{rcases} \)%
}{коэффициенты разложения Ми в дальнем поле соответствующие электрическим и
    магнитным мультиполям}
\nomenclature[a\( e \)]{\( {\boldsymbol{\hat{\mathrm e}}} \)}{единичный вектор}
\nomenclature{\( E_0 \)}{амплитуда падающего поля}
\nomenclature{\( j \)}{тип функции Бесселя}
\nomenclature{\( k \)}{волновой вектор падающей волны}
\nomenclature{%
    \( \begin{rcases}
        a_n \\
        b_n
    \end{rcases} \)%
}{и снова коэффициенты разложения Ми в дальнем поле соответствующие
    электрическим и магнитным мультиполям. Добавлено много текста, так что
    описание группы условных обозначений значительно превысило высоту этой
    группы...}
\nomenclature{\( L \)}{общее число слоёв}
\nomenclature{\( l \)}{номер слоя внутри стратифицированной сферы}
\nomenclature{\( \lambda \)}{длина волны электромагнитного излучения в вакууме}
\nomenclature{\( n \)}{порядок мультиполя}
\nomenclature{%
    \( \begin{rcases}
        {\mathbf{N}}_{e1n}^{(j)} & {\mathbf{N}}_{o1n}^{(j)} \\
        {\mathbf{M}_{o1n}^{(j)}} & {\mathbf{M}_{e1n}^{(j)}}
    \end{rcases} \)%
}{сферические векторные гармоники}
\nomenclature{\( \mu \)}{магнитная проницаемость в вакууме}
\nomenclature{\( r, \theta, \phi \)}{полярные координаты}
\nomenclature{\( \omega \)}{частота падающей волны}

С помощью \verb+nomenclature+ можно включать в~список сокращения,
не~используя их~в~тексте.
% запись сокращения в список происходит командой \nomenclature,
% а не употреблением самого сокращения
\nomenclature{FEM}{finite element method, метод конечных элементов}
\nomenclature{FIT}{finite integration technique, метод конечных интегралов}
\nomenclature{FMM}{fast multipole method, быстрый метод многополюсника}
\nomenclature{FVTD}{finite volume time-domain, метод конечных объёмов
    во~временной области}
\nomenclature{MLFMA}{multilevel fast multipole algorithm, многоуровневый
    быстрый алгоритм многополюсника}
\nomenclature{BEM}{boundary element method, метод граничных элементов}
\nomenclature{CST MWS}{Computer Simulation Technology Microwave Studio
    программа для компьютерного моделирования уравнен Максвелла}
\nomenclature{DDA}{discrete dipole approximation, приближение дискретиных
    диполей}
\nomenclature{FDFD}{finite difference frequency domain, метод конечных
    разностей в~частотной области}
\nomenclature{FDTD}{finite difference time domain, метод конечных разностей
    во~временной области}
\nomenclature{MoM}{method of moments, метод моментов}
\nomenclature{MSTM}{multiple sphere T-Matrix, метод Т-матриц для множества
    сфер}
\nomenclature{PSTD}{pseudospectral time domain method, псевдоспектральный метод
    во~временной области}
\nomenclature{TLM}{transmission line matrix method, метод матриц линий передач}

\FloatBarrier
